\usepackage{xparse}
\usepackage[table]{xcolor}

\newcommand{\uc}{Use Case\xspace}
\newcommand{\ucs}{Use Cases\xspace}

%--------------------------------------------------
%\begin{Actors}
%\Actor
%	{Actor}
%	{Description}
% \\
%\Actor
%	{Actor}
%	{Description}
%\end{Actors}
%--------------------------------------------------
\newcommand{\ActorWidth}{0.15 \textwidth}
\newcommand{\ActorDescWidth}{0.8  \textwidth}

\newenvironment{Actors}
{%
	\rowcolors{1}{gray!20!white}{} % Color on every other line
	\begin{longtable}{p{\ActorWidth} p{\ActorDescWidth}} \hline
	\hiderowcolors % We don't want the color on the title line
	\textbf{Actor} & \textbf{Description} \\ \hline
	\showrowcolors % Show color for all rows to come
}%
{ \hline
	\end{longtable}%
}

\newcommand{\Actor}[2]{#1 & #2 \\}
%--------------------------------------------------



%--------------------------------------------------
%\begin{TestCase}
%\TC
%{Steps}
%{Aktion to take}
%{Expectation}
%{Leave blank}
%\end{TestCase}
%--------------------------------------------------
\newcommand{\TCsteps}{0.05 \textwidth}
\newcommand{\TCexpected}{0.38 \textwidth}
\newcommand{\TCmethod}{0.38 \textwidth}
\newcommand{\TCsucces}{0.08 \textwidth}

\newenvironment{TestCase}
{\begin{longtable}{ p{\TCsteps}  p{\TCexpected} p{\TCmethod} p{\TCsucces} }
\hline \textbf{Trin} & \textbf{Metode} & \textbf{Forventet} & \textbf{Succes}}
{\\ \hline \end{longtable}}

\newcommand{\TC}[4]{\\ \hline \parbox[t]{\TCsteps}{#1} & \parbox[t]{\TCmethod}{#2} &\parbox[t]{\TCexpected}{#3}&  \parbox[t]{\TCsucces}{#4} }
%--------------------------------------------------



%--------------------------------------------------
%\begin{UseCases}
%\UC
%{Use Case Name}
%{Description}
%\UC
%{Use Case Name}
%{Description}
%\end{UseCases}
%--------------------------------------------------
\newcommand{\UCNamesteps}{0.25 \textwidth}
\newcommand{\UCDescription}{0.7 \textwidth}

\newenvironment{UseCases}
{%
	\rowcolors{1}{gray!20!white}{} % Row color shifting
	\begin{longtable}{ p{\UCNamesteps}  p{\UCDescription} }%
	\hline %
	\hiderowcolors % We don't want the color on the title line
	\textbf{Use Case} & \textbf{Description} \\ \hline %
	\showrowcolors % Show color for all rows to come	
}
{\hline \end{longtable}}

\newcommand{\UC}[2]{#1 & #2 \\}


%--------------------------------------------------